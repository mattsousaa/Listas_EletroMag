\documentclass[11pt,a4paper]{article}

\usepackage{epsfig}
\usepackage{multicol}

\usepackage[utf8]{inputenc}
\usepackage[brazil]{babel}
\usepackage{fancyheadings}
\usepackage{amsmath}
\usepackage{calrsfs}
\usepackage{enumerate}
\usepackage{enumitem}   
\DeclareGraphicsExtensions{.png,.pdf}
\usepackage{amsmath, amsfonts, amssymb}
\usepackage{esint}
\usepackage{graphicx}
\usepackage{multicol}
\usepackage{tasks}
\usepackage[utf8]{inputenc}
\usepackage{mathrsfs} % Transformada de Laplace
\usepackage{indentfirst}
\usepackage{xcolor}

% As margens
\setlength{\textheight}{24.0cm}
\setlength{\textwidth}{17.5cm}
\setlength{\oddsidemargin}{2.0cm} % Margens reais desejadas
\setlength{\evensidemargin}{2.0cm} % 2+17.5+1.5=21cm (largura A4)
\setlength{\topmargin}{1.5cm} % 1.5+1.6+1.0+24.0+1.6=29.7cm
\setlength{\headheight}{1.6cm} % (altura A4)
\setlength{\headsep}{1.0cm}
\setlength{\columnsep}{1.5cm} % Coluna = 8cm ((17.5-1.5)/2)
\addtolength{\oddsidemargin}{-1in}
\addtolength{\evensidemargin}{-1in}
\addtolength{\topmargin}{-1in}
\setlength{\footskip}{0.0cm}


% Novos comandos
\newcommand{\limite}{\displaystyle\lim}
\newcommand{\integral}{\displaystyle\int}
\newcommand{\somatorio}{\displaystyle\sum}
\newcommand{\mat}[1]{\mbox{\boldmath{$#1$}}} 

\pagestyle{fancy}


\usepackage{lipsum}

\lhead{
\includegraphics[width=1cm]{brasao.png}
}

\rhead{ 
\sc\textbf{U}niversidade \textbf{F}ederal do \textbf{C}eará\\
Campus Quixadá\\ Lista 7 de Eletromagnetismo}

\cfoot{}

\begin{document}

	\begin{center}
		\Large Magnetostática - Campo e Força magnética. 
	\end{center}

\begin{flushleft}
\textbf{Nome:} Mateus Sousa Araújo. \\
\textbf{Matrícula:} 374858. \\
\textbf{Professor:} Antônio Joel Ramiro de Castro. \\
\textbf{Curso:} Engenharia de Computação. \\
\end{flushleft}

\begin{enumerate}

\item \textbf{Griffiths - Cap. 5 - Problema 5.3.}

Em 1887 J.J. Thomson 'descobriu' o elétron medindo a razão carga-por-massa de uma partícula de 'raios catódicos' (na realidade, feixe de elétrons, com carga $q$ e massa $m$) como se segue:

\begin{enumerate}
\item Primeiro ele passou o feixe através de campos cruzados $E$ e $B$ (mutualmente perpendiculares e ambos perpendiculares ao feixe), e foi ajustado o campo elétrico até atingir deflexão zero. Qual seria, então, a velocidade das partículas (em termos de $E$ e $B$)?

\item Ele, então, desligou o campo elétrico e mediu o raio de curvatura, $R$, do feixe, sujeito apenas à deflexão do campo magnético. Em termos de $E$, $B$ e $R$, qual a razão entre carga e massa $(q/m)$ das partículas?
\end{enumerate}


\textbf{RESOLUÇÃO}

\begin{enumerate}

\item 

Usando a equação:

$$F = q[E + (v \times B)] = 0$$

temos que:

$$E = vB$$

A velocidade então é:

$$v = \displaystyle\dfrac{E}{B}$$

\item

Da equação:

$$m v = q B R$$

tiramos:

$$\displaystyle\dfrac{q}{m} = \displaystyle\dfrac{v}{BR}$$

Substituindo o que foi encontrado na letra $(a)$, obtemos:

$$\displaystyle\dfrac{q}{m} = \displaystyle\dfrac{E}{B^2R}$$

\end{enumerate}


\item \textbf{Griffiths - Cap. 5 - Problema 5.4.}


Suponha que o campo magnético em uma determinada região tem a forma

$$B = kz \ \hat{x}$$

(onde $k$ é uma constante). Encontre a força em um circuito quadrado (de lado $a$), que está no plano $yz$ e centrado na origem, se ele tem uma corrente $I$, que flui no sentido anti-horário quando se olha de cima do eixo $x$.

\textbf{RESOLUÇÃO}

Supondo que a corrente esteja circulando em algum sentido na espira, podemos dizer que as forças magnéticas se cancelam nas pontas. A força acima da espira é:

$$IaB = Iak(a/2) = Ika^2/2$$

A força acima tem sinal positivo, já que a mesma está apontando para cima. A força para baixo é a mesma em módulo. Podemos concluir então que a força total é:

$$F = Ika^2 \ \hat{z}$$


\item \textbf{Griffiths - Cap.5 - Problema 5.5.}

Uma corrente $I$ flui por um fio de raio $a$.

\begin{enumerate}
\item Se ela estiver distribuída uniformemente sobre a superfície, qual é a densidade superficial de corrente $K$?
\item Se ela estiver distribuída de forma que a corrente volumétrica seja inversamente proporcional à distância do eixo, quanto vale $J$?
\end{enumerate}

\textbf{RESOLUÇÃO}

\begin{enumerate}

\item

$$K = \displaystyle\dfrac{I}{2\pi a}$$

\item

Pela relação:

$$\displaystyle\dfrac{\alpha}{s}$$

temos que a corrente $I$ é:

$$I = \displaystyle\int J \ da = \alpha \displaystyle\int \displaystyle\dfrac{1}{s} \ ds \ d\phi = 2\pi \alpha \displaystyle\int ds = 2 \pi a \alpha$$

Isolando $\alpha$, temos:

$$\alpha = \displaystyle\dfrac{I}{2\pi a}$$

Logo, $J$ é:

$$J = \displaystyle\dfrac{I}{2\pi a}$$

\end{enumerate}


\end{enumerate}
	
\end{document}