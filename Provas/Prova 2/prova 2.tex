\documentclass[11pt,a4paper]{article}

\usepackage{epsfig}
\usepackage{multicol}

\usepackage[utf8]{inputenc}
\usepackage[brazil]{babel}
\usepackage{fancyheadings}
\usepackage{amsmath}
\usepackage{calrsfs}
\usepackage{enumerate}
\usepackage{enumitem}   
\DeclareGraphicsExtensions{.png,.pdf}
\usepackage{amsmath, amsfonts, amssymb}
\usepackage{esint}
\usepackage{graphicx}
\usepackage{multicol}
\usepackage{tasks}
\usepackage[utf8]{inputenc}
\usepackage{mathrsfs} % Transformada de Laplace
\usepackage{indentfirst}
\usepackage{xcolor}

% As margens
\setlength{\textheight}{24.0cm}
\setlength{\textwidth}{17.5cm}
\setlength{\oddsidemargin}{2.0cm} % Margens reais desejadas
\setlength{\evensidemargin}{2.0cm} % 2+17.5+1.5=21cm (largura A4)
\setlength{\topmargin}{1.5cm} % 1.5+1.6+1.0+24.0+1.6=29.7cm
\setlength{\headheight}{1.6cm} % (altura A4)
\setlength{\headsep}{1.0cm}
\setlength{\columnsep}{1.5cm} % Coluna = 8cm ((17.5-1.5)/2)
\addtolength{\oddsidemargin}{-1in}
\addtolength{\evensidemargin}{-1in}
\addtolength{\topmargin}{-1in}
\setlength{\footskip}{0.0cm}


% Novos comandos
\newcommand{\limite}{\displaystyle\lim}
\newcommand{\integral}{\displaystyle\int}
\newcommand{\somatorio}{\displaystyle\sum}
\newcommand{\mat}[1]{\mbox{\boldmath{$#1$}}} 

\pagestyle{fancy}


\usepackage{lipsum}

\lhead{
\includegraphics[width=1cm]{brasao.png}
}

\rhead{ 
\sc\textbf{U}niversidade \textbf{F}ederal do \textbf{C}eará\\
Campus Quixadá\\ Avaliação Parcial 2}

\cfoot{}

\begin{document}

	\begin{center}
		\Large Corrente elétrica, Campo Magnético, Lei de Ampére, Lei da indução, Circuitos e Materiais Magnéticos. 
	\end{center}

\begin{flushleft}
\textbf{Nome:} Mateus Sousa Araújo. \\
\textbf{Matrícula:} 374858. \\
\textbf{Professor:} Antônio Joel Ramiro de Castro. \\
\textbf{Curso:} Engenharia de Computação. \\
\end{flushleft}

\begin{enumerate}

\item \textbf{Moysés Nussenzveig - Cap. 6 - Problema 1.}

Uma válvula diodo da era pré-transistor contém um par de placas planas paralelas de espaçamento $d$, no vácuo. Estabelece-se entre elas uma diferença de potencial $V$. Um feixe de elétrons com área de seção transversal A e de velocidade inicial $v_0$ é emitido a partir de umadas placas (catodo) e acelerado até a outra (anodo), produzindo uma corrente estacionária de intensidade i. 

\begin{enumerate}

\item Calcule a velocidade $v(x)$ de um elétron à distância $x$ do cátodo.
\item Calcule a densidade $n(x)$ de elétrons no feixe como função de x. Suponha que $i$ é suficientemente fraco para que o campo gerado pelos elétrons seja desprezível em confronto com o campo acelerador.
\end{enumerate}

\textbf{RESOLUÇÃO}

\begin{enumerate}

\item 

Podemos considerar inicialmente a 2ª lei de Newton com uma força resultante de natureza elétrica. Esse campo é atuante entre as duas placas paralelas, logo podemos escrever as seguintes relações:

$$F = q E = m a$$ 

$$q V/d = m \displaystyle\dfrac{V^2 - V_0^2}{2x}$$ 

Isolando $V(x)$ temos:

$$V(x) = \left(\displaystyle\dfrac{2qV_x}{mV_0^2d} + 1\right)V_0$$

\item

A densidade de corrente elétrica é dada por:

$$J = n q V$$

Sabemos que $J = i/A$. Substituindo na equação acima, temos:

$$n = \displaystyle\dfrac{J}{q V} = \displaystyle\dfrac{i}{q A V} = \displaystyle\dfrac{i}{e A J_0 } \left(\displaystyle\dfrac{2qV_x}{mV_0^2d} + 1\right)^{-1/2}$$
\end{enumerate}


\item \textbf{Moysés Nussenzveig - Cap. 6 - Problema 4.}

O campo elétrico médio na atmosfera, perto da superfície terrestre, é de 100 $V/m$ dirigido para a Terra. A corrente média de íons que atinge a totalidade da superfície da terra é de 1800 $A$. Supondo que a distribuição da corrente é isotrópica, calcule a condutividade do ar na vizinhança da superfície da Terra.


\textbf{RESOLUÇÃO}

Considerando uma distribuição superficial de cargas em termos da densidade de corrente, temos:

$$J = \sigma E$$

Podemos também escrever que a densidade de corrente é:

$$J = i/S$$

Igualando as duas, temos:

$$\sigma E = \displaystyle\dfrac{i}{S} = \displaystyle\dfrac{i}{S E}$$

$$\sigma = \displaystyle\dfrac{1200}{4 \pi (6,37 \cdot 10^6)^2 \ 100}$$

$$\sigma = 3,53 \cdot 10^{-14} \ \Omega \ m^{-1}$$

Substituindo os valores da área de uma esfera e colocando os valores correspondentes temos:


\item \textbf{Moysés Nussenzveig - Cap. 7 - Problema 1.}

Uma bússola tende a oscilar antes de alinhar-se com o campo magnético da Terra. Considere uma agulha imantada de momento de dipolo magnético $m$ e momento de inércia $I$, suspensa de forma a poder oscilar livremente em torno de um eixo vertical, situada num campo magnético horizontal uniforme $B_0$. As direções de $m$ e $B_0$ formam inicialmente um pequeno ângulo $\theta_0$. Calcule a frequência angular de oscilação (desprezando o amortecimento) e mostre que sua determinação permite $|m| \cdot |B_0|$.

\textbf{RESOLUÇÃO}

\begin{figure}[h]	
\centering % para centralizarmos a figura	
\includegraphics[width=5cm]{Selection_098.jpg} 
\end{figure}

Podemos generalizar as grandezas dadas no problema e plotar em um plano. As grandezas $m$ e $B_0$ permitem que nós achemos um outro vetor resultante da operação vetorial dos dois. Esse vetor resultante é uma força de tração, que tende a colocar um corpo no seu estado inicial de repouso. Podemos escrever, então:

$$T = m \times B_0 S$$ 
$$T = m B_0 \sin \theta$$

Para valores pequenos de $\theta$, temos senos pequenos. Isso é muito semelhante a relação da lei de Hooke para fios:

$$T = m B_0 \theta$$

$$T = k\theta$$

A torção $T$ é proporcional ao ângulo dos vetores formados. Podemos considerar um movimento harmônico simples (MHS) com frequência $\omega = 2\pi/T$. Se a aceleração de um sistema massa-mola simples é $\alpha = \omega^2x$, podemos substituir na 2ª lei de Newton:

$$F = m \omega^2 x$$

Como $m$ e $\omega$ são grandezas constantes dentro do MHS, podemos expressar:

$$k = m\omega^2$$

Isolando $\omega$, temos a frequência angular:

$$\omega = \sqrt{\displaystyle\dfrac{k}{m}}$$

Podemos escrever agora que:

$$\omega = \sqrt{\displaystyle\dfrac{m B}{I}}$$

\item \textbf{Moysés Nussenzveig - Cap. 7 - Problema 3.}

\begin{enumerate}
\item Calcule a frequência angular de rotação de um elétron no campo magnético da terra, numa região em que ele possa ser tratado como uniforme e de intensidade $0,5$ Gauss. 
\item Para um elétron com energia cinética de 1 $keV$, típica daquela encontrada na aurora boreal, calcule o raio de curvatura nesse campo.
\end{enumerate}



\textbf{RESOLUÇÃO}

\begin{enumerate}

\item Podemos igualar a força magnética a força centrípeta. Dessa forma, temos:

$$F_b = qvB = \displaystyle\dfrac{mv^2}{R} = Bq = \displaystyle\dfrac{mv}{R} = m \omega$$

$$\omega = \displaystyle\dfrac{Bq}{m}$$

Transformando o campo magnético de Gauss em Tesla temos $B = 5 \cdot 10^{-5}$ T. Considerando o valor da carga elétrica e a massa de um elétron, temos:

$$\omega = \displaystyle\dfrac{5 \cdot 10^{-5} \ 1,6 \cdot 10^{-19}}{9,1 \cdot 10^{-31}}$$

$$\omega = 8,79 \cdot 10^{6} \ \textrm{rad}$$ 

Calculando a frequência usando o resultado obtido acima, temos:

$$f = \displaystyle\dfrac{\omega}{2\pi} = \displaystyle\dfrac{8,79 \cdot 10^6}{2\pi} = 1,4 \cdot 10^6 \ \textrm{Hz}$$

\item Colocando na relação de energia cinética temos:

$$E = \displaystyle\dfrac{1}{2}\displaystyle\dfrac{m}{v^2}$$

$$v = \sqrt{\displaystyle\dfrac{2E}{m}}$$

$$v = \sqrt{\displaystyle\dfrac{2.1000.1,6 \cdot 10^{-14}}{9,1 \cdot 10^{-31}}} = 18,75 \cdot 10^6 \textrm{m/s}$$

Como temos:

$$r = \displaystyle\dfrac{mv}{qB}$$

Substituindo os valores dados na questão, temos que $r = 2,1$ m.

\end{enumerate}

\item \textbf{Moysés Nussenzveig - Cap. 8 - Problema 4.}

Uma espira quadrada de lado $L$ é percorrida por uma corrente $i$. 

\begin{enumerate}
\item Determine, em módulo, direção e sentido, o campo $B$ num ponto $P$ situado sobre o eixo da espira (reta perpendicular ao seu plano passando pelo centro $O$ da espira), à distância $z$ de $O$. 
\item Interprete o resultado obtido para $z >> L$.
\end{enumerate}


\textbf{RESOLUÇÃO}

\begin{enumerate}

\item 


\begin{figure}[h]	
\centering % para centralizarmos a figura	
\includegraphics[width=5cm]{Selection_099.jpg} 
\end{figure}

Podemos calcular o campo produzido apenas por um fio e depois multiplicar por 4. Utilizando a lei de Biot-Savart, temos:

$$B = \displaystyle\dfrac{\mu_0 i}{4 \pi} a \displaystyle\int_0^{L/2} \displaystyle\dfrac{dy}{(a^2 + y^2 + z^2)^{3/2}}$$

Resolvendo a integral acima por substituição trigonométrica obtemos:

$$B = \displaystyle\dfrac{\mu_0 i L^2}{2 \pi \left(\displaystyle\dfrac{L^2}{4} + z^2\right)\sqrt{\displaystyle\dfrac{L^2}{2} + z^2}}$$

\item

Para $z >> L$ temos:

$$B = \displaystyle\dfrac{\mu_0 i S}{2 \pi z^3}$$

\end{enumerate}


\item \textbf{Moysés Nussenzveig - Cap. 9 - Problema 1.}

O princípio do fluxômetro, empregado para medir a intensidade $B$ de um campo magnético, consiste em empregar uma pequena bobina de prova, com $N$ espiras de área $S$, cujos terminais estão ligados a um galvanômetro balístico. A bobina, cuja resistência é $R$, é colocada com o plano das espiras perpendicular ao campo magnético que se deseja medir, do qual é removida subitamente. Isso gera um pulso de corrente, e o galvanômetro balístico mede a carga total $Q$ associada a este pulso. Calcule o valor de $B$ em função de $N$, $S$, $R$ e $Q$.

\textbf{RESOLUÇÃO}

O fluxo que atravessa uma espira de área $S$ com um campo magnético constante $B$ é dado por:

$$\phi = \displaystyle\oint B \cdot ds = B s$$

Generalizando o resultado para $n$ espiras, temos:

$$\phi = n B s$$

Relacionando o fluxo com a carga $q$ temos:

$$i = \displaystyle\dfrac{1}{R}\displaystyle\dfrac{d\phi}{dt}$$

$$i = \displaystyle\dfrac{dq}{dt} = \displaystyle\dfrac{d\phi}{dt}\displaystyle\dfrac{1}{R}$$

Dessa forma, obtemos:

$$\phi = qR$$

Podemos substituir a relação acima em $\phi = n B s$.

$$\phi = n B s = q R$$

$$B = \displaystyle\dfrac{qR}{NS}$$


\item \textbf{Moysés Nussenzveig - Cap. 9 - Problema 2.}

Liga-se um voltímetro entre os trilhos de uma estrada de ferro, cujo espaçamento é de $1,5$ m. Os trilhos são supostos isolados um do outro. A componente vertical do campo magnético terrestre no local é de $0,5$ G. Qual é a leitura do voltímetro quando passa um trem a $150$ km/h?

\textbf{RESOLUÇÃO}

\begin{figure}[h]	
\centering % para centralizarmos a figura	
\includegraphics[width=7cm]{Selection_100.jpg} 
\end{figure}

O fluxo que atravessa o trilho, é dado por:

$$\phi = \displaystyle\oint B \cdot ds $$

Nesse caso o campo é constante e $s$ varia apenas no comprimento. Logo, podemos escrever:

$$\phi = \displaystyle\oint B \cdot ds = B \displaystyle\int_0^x h \ dx = Bhx$$

Esse fluxo pela lei de Faraday, provocará uma força eletromotriz dada por:

$$\varepsilon = -\displaystyle\dfrac{d\phi}{dt} = -\displaystyle\dfrac{d}{dt}(Bhx)$$

Como apenas $B$ e $h$ são fixos então apenas o comprimento varia. Podemos escrever:

$$\varepsilon = -Bhv$$

Substituindo os valores do problema na relação acima obtemos:

$$\varepsilon = 3,13 \cdot 10^{-3} \ \textrm{V}$$

\item \textbf{Moysés Nussenzveig - Cap. 9 - Problema 6.}

Uma espira circular de raio $a$ tem no seu centro uma outra espira circular de raio $b << a$. Os planos das duas espiras formam entre si um ângulo $\theta$. Calcule a indutância mútua entre elas.

\textbf{RESOLUÇÃO}

O fluxo causado entre as espiras, é:

$$\phi_{b,a} = \displaystyle\oint B \cdot ds = \displaystyle\int B ds \cos \theta$$

$$\phi_{b,a} = B \cos \theta \displaystyle\int ds = B \cos \theta \pi b^2$$

Colocando na fórmula do campo em uma espira, temos:

$$B = \displaystyle\dfrac{\mu_0 i}{2a} = \displaystyle\dfrac{\mu_0 i \pi b^2 \cos \theta}{2a}$$

Agora colocando na indutância mútua, obtemos:

$$L = \displaystyle\dfrac{\phi}{i} = \displaystyle\dfrac{\mu_0 \pi b^2 \cos \theta}{2a}$$

\item \textbf{Moysés Nussenzveig - Cap. 10 - Problema 1.}

\begin{figure}[h]	
\centering % para centralizarmos a figura	
\includegraphics[width=5cm]{Selection_101.jpg} 
\end{figure}

No circuito da figura, $R_1 = 20 \Omega$ e $R_2 = 60 \Omega$. Para que valor de R a potência dissipada em R é afetada o mínimo possível por pequenas variações de R?

\textbf{RESOLUÇÃO}

Fazendo as leis das malhas em ambas malhas no circuito temos:

$$
\begin{cases}
\varepsilon - 20 i_1 - 60 i_2 = 0 \\
-60 i_2 + R i_3 = 0 \\

\end{cases}
$$

Sabemos também que $i_2 = i_1 - i_3$. Substituindo na primeira equação temos:

$$\varepsilon - 20 (i_2 + i_3) - 60 i_2 = 0$$

$$\varepsilon - 80 i_2 - 20 i_3 = 0$$

$$\varepsilon - 80/60 R i_3 - 20 i_3 = 0$$

$$i_3 = \left( 20 + \displaystyle\dfrac{4}{3} R\right) = \varepsilon$$

$$i_3 = \displaystyle\dfrac{\varepsilon}{\left( 20 + \displaystyle\dfrac{4}{3} R\right)}$$

A potência dissipada no resistor vai ser:

$$P = R i_3^2 = \displaystyle\dfrac{R \varepsilon^2}{\left( 20 + \displaystyle\dfrac{4}{3} R\right)^2}$$

Pegando a expressão que obtivemos, basta derivar em relação a R e igualar a zero. 

$$\displaystyle\dfrac{dP}{dR} = 0$$

Fazendo algebrismos, encontramos que a resistência mínima para menor potência no circuito é:
$$R = 15 \ \Omega$$.

\item \textbf{Moysés Nussenzveig - Cap. 11 - Problema 6.}

No circuito magnético da figura, a secção reta é constante, a permeabilidade magnética do material é $\mu$ e a corrente na bobina de $N$ espiras é $i$. Calcule o campo $B_1$ no braço central e o campo $B_2$ nos demais braços. 

\textbf{RESOLUÇÃO}

Usando a Lei de Ampére e envolvendo um dos braços de um transformador simples de $N$ espiras e com corrente $I$, temos:

$$\displaystyle\oint H \cdot dl = NI$$

Se o campo for constante, temos:

$$H = \displaystyle\dfrac{NI}{L}$$

Basta saber como se comporta $H$ nos braços do transformador e no centro. Para isso, basta saber o fluxo do campo em cada parte. Para o braço esquerdo, temos:

$$H_{esq} = \displaystyle\dfrac{NI}{2a + 2b}$$

Para o braço direito, temos:

$$H_{dir} = \displaystyle\dfrac{NI}{2a + 2b}$$

Para o centro temos:

$$H_{cent} = \displaystyle\dfrac{NI}{b}$$

Agora basta colocar na fórmula $B = \mu H$ para saber o campo. O campo no braço esquerdo é:

$$B_{esq} = \displaystyle\dfrac{\mu NI}{2a + 2b} \ T$$

O campo no braço direito é:

$$B_{dir} = \displaystyle\dfrac{\mu NI}{2a + 2b} \ T$$

O campo no centro é:

$$B_{cent} = \displaystyle\dfrac{\mu NI}{b} \ T$$

\begin{figure}[h]	
\centering % para centralizarmos a figura	
\includegraphics[width=6cm]{Selection_102.jpg} 
\end{figure}


















\end{enumerate}
	
\end{document}